\documentclass[a4paper,12pt]{report}
%\usepackage[top=2.5cm, bottom=2.5cm, left=2.5cm, right=2.5cm]{geometry}
\usepackage[utf8x]{inputenc}
\usepackage[T1]{fontenc}
\usepackage[french]{babel} 
\usepackage{graphicx}
\usepackage{hyperref}

\author{\textsc{Jorandon} Guillaume, \textsc{Marcoux} Vincent\\\small{Responsable pédagoqique : \textsc{Collomb} Cléo}}
\date{22 mai 2017} 
\title{\includegraphics[width=140px]{utc.jpg}\\\vspace{25mm}SC22 P17\\\normalsize{Discriminations et algorithmes}}

\begin{document}

\hypersetup{pageanchor=false}
\maketitle
\tableofcontents
\hypersetup{pageanchor=true}
\chapter*{Introduction}
\addcontentsline{toc}{chapter}{Introduction}
Nous vivons aujourd'hui dans un monde où l'évolution technologique est très rapide, et où on ne peut plus ne pas avoir entendu parler des algorithmes, nouveaux outils omniprésents. Dans la société qu'est la notre, hyper-technologique, il est en effet difficile de les éviter, et ils nous apparaissent comme des outils très utiles et faciles à utiliser. Ces algorithmes sont pour partie le fait de grands groupes qui recherchent le profit, et qui dans cet objectif exploitent des corrélations statistiques pour générer de l'information exploitable. En faisant passer un grand volume de données connectées au travers d'algorithmes toujours plus efficaces, ces acteurs industriels majeurs peuvent faire du profilage sur la population : c'est ce qu'on appelle la fouille de données, enseignée à l'UTC dans la filière FDD du Génie Informatique, et que l'on rassemble parfois sous le terme vague de \textit{Big Data}. Ce savoir constitue un ``pouvoir statistique'' énorme qui confère aux entreprises qui le possèdent une force de frappe politique et économique considérable, et est source d'une certaine forme de gouvernance, la gouvernementalité algorithmique. Comme toute source de pouvoir, il convient alors de la questionner et de se demander dans quel mesure elle est légitime. Peut-elle être discriminatoire, subjective, guidée par une certaine idéologie, ou est-elle neutre et objective ? Sur qui ou quoi faire reposer la responsabilité des dérives de cette gouvernementalité algorithmique ? Mais surtout, peut-on parler de discrimination algorithmique ? 

Dans le cadre de l'UV SC22, nous avons étudié une approche culturelle des techniques, notamment au travers des prismes de genre et de classe. Dans ce cadre, nous avons étudié l'origine de certains mécanismes systémiques, sources d'oppressions et de discriminations : sexistes, raciales, sur l'identité sexuelle et de genre, etc. En nous appuyant sur le cours mais aussi sur l'étude de plusieurs articles, nous allons essayer de définir ce qu'est la gouvernementalité algorithmique et son fonctionnement. Nous verrons dans quelles mesures les discriminations algorithmiques s'inscrivent dans cette gouvernementalité. Nous étudierons plusieurs exemples pour illustrer nos propos.

\appendix
\bibliographystyle{alpha}
\bibliography{rapport}
\addcontentsline{toc}{chapter}{Bibliographie}

\end{document}